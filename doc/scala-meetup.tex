% 

\documentclass{beamer}
\usepackage[utf8]{inputenc}
\usepackage{marvosym} % \MVRIGHTarrow
\usepackage{hyperref}
\begin{document}
\title{Probabilistisches Programmieren in Scala}   
\author{Christoph Schmalhofer} 
\date{2018-09-27}
\frame{\titlepage} 


\section{} 
\subsection{}
\frame{\frametitle{Lernfähige Simulationen I}
\begin{itemize}
\item Reconstructing Fusion Plasmas
\item Network Processing Vertically Integrated Seismic Analysis
\item Improvements to Inference Compilation for
Probabilistic Programming in
Large-Scale Scientific Simulators
\end{itemize} 
}


\section{} 
\subsection{}
\frame{\frametitle{Lernfähige Simulationen II}
\begin{itemize}
\item Vorwissen: Regeln/Struktur und Verteilung Variablenwerte
\item Daten
\item Inferenz \MVRightarrow{}  mehr Wissen über Variablenwerte \MVRightarrow{}  bessere Vorhersagen
\end{itemize} 
}


\section{} 
\subsection{}
\frame{\frametitle{Fixpunkt}
\begin{itemize}
\item Naive Bayes, Markov Kette
\item Bayes Netz, Gauß Prozess
\item sind alles Probabilistische Programme
\end{itemize} 
}



\section{} 
\subsection{}
\frame{\frametitle{Sprachen/Bibliotheken}
\begin{itemize}
\item Bugs, Jags, Stan
\item Hansei, Haraku, Haraku10
\item PyMC3, Pyro, Tensorflow Probability
\item Figaro, Rainier
\item ...
\end{itemize} 
}



\section{} 
\subsection{}
\frame{\frametitle{einfaches Modell für die Code Beispiele}
\begin{itemize}
\item abhängig von einem Münzwurf 
\item Kopf \MVRightarrow{}  Anzahl Kopf bei zwei weiteren Würfen
\item Zahl \MVRightarrow{}  Anzahl Kopf bei drei weiteren Würfen
\end{itemize} 
}

\section{} 
\subsection{}
\frame{\frametitle{Wahrscheinlichkeitsmonade}
\url{https://github.com/probprob/probabilisticprogramming/tree/master/scala/src/main/scala}
}

\section{} 
\subsection{}
\frame{\frametitle{Figaro}
\url{https://github.com/probprob/probabilisticprogramming/tree/master/scala/src/main/scala/figaro}
}


\section{} 
\subsection{}
\frame{\frametitle{Rainier}
\url{https://github.com/probprob/probabilisticprogramming/tree/master/scala/src/main/scala/rainier}
}


\end{document}


